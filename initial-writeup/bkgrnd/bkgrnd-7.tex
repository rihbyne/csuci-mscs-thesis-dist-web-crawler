% background chapter continued

\section{Amazon Web Services}\index{Amazon Web Services}
\subsection{Identity \& Access Management (IAM)}
\subsection{Virtual Private Cloud (VPC)}\index{Virtual Private Cloud (VPC)}
\subsection{Internet Gateway (IGW)}
\subsection{Network Address Translation (NAT)}
\subsection{Elastic Compute Cloud (EC2)}\index{Elastic Compute Cloud (EC2)}
\subsection{Relational Database Service (RDS)}
\subsection{Elastic Cache}
\subsection{MongoDB}

\pagebreak

\subsection{AWS Cloudformation}
According to aws documentation \cite{docker}, cloudformation(CF) is a service which lets you model and set up
aws resources so that you spend less time managing resources and focus more on application logic that runs
on those resources. Basically, in a template you describe all resources that make your architecture. You
dont need to individually create and configure aws resources and figure what what's dependent on what.
\\
\\
some scenarios that demonstrate how aws cloudformation can help.
\begin{itemize}
    \item simplify infrastructure management
    \item quickly replicate infrastructure
    \item easily control and track changes in your infrastructure
\end{itemize}
\\
Some of the big concepts in CF are \textit{Templates}, \textit{Stacks}, and \textit{ChangeSets}.
\begin{itemize}
\item \textbf{Templates:} CF uses templates saved in TXT, YML, or JSON as a blueprint for building AWS
  resources. The templates can be reused on altogether different regions by supplying input parameters at the time of stack creation.
\item \textbf{Stacks:} CF manages related resources as a single unit called a stack. Creating, updating,
  and deleting a collection of resources is done by creating, updating, and deleting stacks. All the
  resources in a stack are defined by the specified CF template. To create resources which involves 
  - Auto-Scaling group, Elastic Load Balancer(ELB), and an Amazon RDS database instance, create a stack by
  submitting the template consisting of those resources and CF provisions all those resources on your
  behalf. This requires a service role to be present before CF can actually create resources, refer to
  section \ref{iamroles} under whirlpool operations. To addon, Stacks can be operated using AWS CF console, APIs, and AWS CLI.
\item \textbf{ChangeSets:} Making changes to the running resources in a stack, updates the stacks. Before
  changes are applied to the resources, a change set is generated which is a summary of proposed changes.
  ChangeSets makes the operator see how the his/her changes might impact current running resources,
  especially for critical resources, before implementing them. For example, if the name of the db
  identifier of Amazon RDS instance is changed, CF will create a new database and delete the one one.
  Data in the old database gets lost unless it was already backed up. If the given changeset is generated,
  operator will see that the change will cause database to be replaced and therefore the operator will be
  able to plan accordingly before update to a stack is applied.
\end{itemize}

\pagebreak

\subsection{AWS Monthly Calculator}

\pagebreak
\section{Docker Containers}\label{dockerintro}
A docker container along with docker-engine is something. 

\subsection{DockerFile}\index{DockerFile}
(place a sample of the file here)
\begin{itemize}
\item RUN is an image build step, the state of the container after a RUN command will be committed to the docker image.
  A Dockerfile can have many RUN steps that layer on top of one another to build the image. Note than each RUN command
  will create a new layer and therefore affects the size of the container. It is advisable to club together unix command
  in a single RUN statement to avoid unnecessary container size increase.
\item CMD is the command the container executes by default when you launch the built image.
  A Dockerfile can only have one CMD. The CMD can be overridden when starting a container with docker run
  `image` other_command.
\item ENTRYPOINT is also closely related to CMD and can modify the way a container starts an image.
\item VolUMES useful to map a directory from container to host. Can be specified explicitly or using docker
  volumes - docker volume create 'sample-volume' and setting external volume to true.
\item ARG & ENV
\item FROM base-image as some-other-image used to specify targets to package a specific build
\item external networks
\end{itemize}


\subsection{Docker Compose}\index{Docker Compose}
docker-compose is installed separately and is optional
% background chapter continued

\section{Amazon Web Services(AWS)}
AWS is an on-demand cloud computing platform delivering technical infrastructure to customers through its
web services. The customers are billed on a metered pay-as-you-go basis. Following is a short summary of
list of services applied to run the whirlpool web crawler.

\begin{description}
  \item[Identity \& Access Management (IAM)] \hfill \\
    IAM is a free of charge access control service. Allows aws account holder to create and manage AWS
    users groups, and roles. Also enforce permissions on them to allow/deny access to AWS resources
    by attaching to it managed and custom policies.
  \item[Virtual Private Cloud (VPC)] \hfill \\
    VPC is a logically isolated virtual network environment that solutions architect provisions within
    which you can launch other AWS resources. A VPC is typically composed of public and private subnets.
    You can attach route tables to faciliate communication between different subnets and add a combination
    of stateful/stateless firewalls.

  \item[Internet Gateway (IGW)] \hfill \\
    IGW is a VPC component which is attached to the VPC that establishes connection between your running
    instances in the VPC and the internet. IGW supports IPv4 and IPv6 addresses.

  \item[Network Address Translation (NAT)] \hfill \\
    NAT component has its place in public subnet of the VPC to enable instances in the private subnet
    to initiate outbound traffic to the internet but deny inbound traffic initiated by external
    machine on the internet. There exist NAT gateway and NAT instance. Both serve same purpose but differ
    by features and how they are handled. Just to note, NAT gateway is managed service by AWS whereas
    NAT instance is managed entirely by entity using it.

  \item[Elastic Compute Cloud (EC2)] \hfill \\
    EC2 is most widely used, highly available compute capacity of AWS delivered through its own web
    service. Its the first service of AWS that changed the economics of rental computer hardware by
    allowing customers to pay only for what they used and how long they used.

  \item[{Relational Database Service (RDS)] \hfill \\
      RDS is a managed RDMS service offered by AWS. It is also called database-as-a-Service(DaaS), mainly
      because the cost of hardware provisioning, maintenance, scaling, backups, and patching is taken
      care by external vendor, in this case, the AWS increasing productivity of users with no special
      expertise.

  \item[{Elastic Cache] \hfill \\
      Elastic-cache is a in-memory service offered by AWS. Its a managed instance of redis or memcached
      in-memory data stores. Cache is used for improving performance of data intensive applications.
      AWS abstracts away the underpinnings of achieving highly-available, fault tolerant caching mechanism.


  \item[{MongoDB] \hfill \\
      MongoDB is a variant of NoSQL Database. It is a document oriented database. Documents reside in
      collections. It has schemaless structure which makes it suitable for storing unstructured data.
      It is also built with a purpose to easily achieve horizontal scalability.
\end{description}


\pagebreak

\subsection{AWS Cloudformation}
According to aws documentation \cite{docker}, cloudformation(CF) is a service which lets you model and set up aws resources so that you spend less time managing resources and focus more on application logic that runs on those resources. Basically, in a template you describe all resources that make your architecture. You
dont need to individually create and configure aws resources and figure what what's dependent on what.
\\
\\
some scenarios that demonstrate how aws cloudformation can help.
\begin{itemize}
    \item simplify infrastructure management
    \item quickly replicate infrastructure
    \item easily control and track changes in your infrastructure
\end{itemize}
\\
Some of the big concepts in CF are \textit{Templates}, \textit{Stacks}, and \textit{ChangeSets}.
\begin{itemize}
\item \textbf{Templates:} CF uses templates saved in TXT, YML, or JSON as a blueprint for building AWS
  resources. The templates can be reused on altogether different regions by supplying input parameters at the time of stack creation.
\item \textbf{Stacks:} CF manages related resources as a single unit called a stack. Creating, updating,
  and deleting a collection of resources is done by creating, updating, and deleting stacks. All the
  resources in a stack are defined by the specified CF template. To create resources which involves 
  - Auto-Scaling group, Elastic Load Balancer(ELB), and an Amazon RDS database instance, create a stack by
  submitting the template consisting of those resources and CF provisions all those resources on your
  behalf. This requires a service role to be present before CF can actually create resources, refer to
  section \ref{iamroles} under whirlpool operations. To addon, Stacks can be operated using AWS CF console, APIs, and AWS CLI.
\item \textbf{ChangeSets:} Making changes to the running resources in a stack, updates the stacks. Before
  changes are applied to the resources, a change set is generated which is a summary of proposed changes.
  ChangeSets makes the operator see how the his/her changes might impact current running resources,
  especially for critical resources, before implementing them. For example, if the name of the db
  identifier of Amazon RDS instance is changed, CF will create a new database and delete the one one.
  Data in the old database gets lost unless it was already backed up. If the given changeset is generated,
  operator will see that the change will cause database to be replaced and therefore the operator will be
  able to plan accordingly before update to a stack is applied.
\end{itemize}

\pagebreak

\subsection{AWS Monthly Calculator}
This is a tool allowing customers to estimate monthly charges against their use of AWS services. This helps
organizations identify areas where it can cut cost and plan according to their budget.

\pagebreak
\section{Docker Containers}\label{dockerintro}
Containers are used to develop and build apps once, locally and run anywhere. Being flexible and lightweight, they are used to containerized complex programs such as this. Docker has its own terminology and concepts to be able to use it correctly. So a
docker image is a executable package that includes everything required to run the
program. A running instance of an image is called a container.

\subsection{DockerFile}
\inputminted[
fontfamily=tt,
baselinestretch=1.2,
fontsize=\footnotesize,
linenos,
numbersep=5pt,
tabsize=2,
frame=single]{yaml}{../../whirlpool/crawler/whirlpool-urlfilter/Dockerfile}

\noindent
This is a dockerfile for \texttt{whirlpool-parser} docker image, it defines what goes on in the environment inside your container.

\pagebreak

\subsection{Docker Compose}\index{Docker Compose}
docker-compose is not bundled with docker utility and therefore installed
separately. Compose is a tool for defining multi-container applications. For
e.g, rest api, database server, redis cache are the services that make up
application stack and can be launched with single compose command.

\begin{minted}[]{yaml}
  version: '3'
  services:
    rest-api:
      build: .
      ports:
      - "5000:5000"
      volumes:
      - .:/code
      - logvolume01:/var/log
      links:
      - redis
    redis:
      image: redis
  volumes:
    logvolume01: {}
\end{minted}